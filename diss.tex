%%%%%%%%%%%%%%%%% EINSTELLUNGEN %%%%%%%%%%%%%%%
\documentclass[a4paper,11pt]{memoir}
\usepackage[no-math]{fontspec}
\usepackage[greek,latin,english,french,german,strings=unicode]{babel}
\languageattribute{greek}{polutoniko}
\newenvironment{griechisch}
{\begin{otherlanguage*}{greek}}%
{\end{otherlanguage*}}
\setmainfont{Noto Serif}
\setsansfont{Noto Sans}
\newfontfamily\hebrewfont[Script=Hebrew]{SBL BibLit}
\usepackage[french=guillemets,german=guillemets]{csquotes} % Verwaltung von Anführungszeichen
\usepackage[backend=biber,style=authortitle-dw,hyperref,edsuper=true,idembib=true,idembibformat=dash,maxnames=4,citetracker=strict,ibidtracker=false,pagetracker=true,terselos=true]{biblatex} % Literaturverwaltung
\bibliography{diss} % Bibtex-Datenbank
\usepackage{microtype} % optischer Randausgleich
%%%%%%%%%%%% Seitenlayout %%%%%%%%%%%%%%%%%%%%%%%%%%
\settrimmedsize{297mm}{210mm}{*}
\setlength{\trimtop}{0pt}
\setlength{\trimedge}{\stockwidth}
\addtolength{\trimedge}{-\paperwidth}
\settypeblocksize{42\onelineskip}{30pc}{*}
\setlrmargins{3.1cm}{*}{*}
\setulmargins{3.5cm}{*}{*}
\setheadfoot{\onelineskip}{3\onelineskip}
\setheaderspaces{*}{\onelineskip}{*}
\setmarginnotes{17pt}{100pt}{\onelineskip}
\checkandfixthelayout
% Kopf-/Fußzeilen
\nouppercaseheads 
\addtopsmarks{headings}{}{
\createmark{chapter}{both}{shownumber}{}{. \space}
}
\pagestyle{headings}
%%%%%%%% Überschriften %%%%%%%%%%%%%%
%%%%%%%%%%%% Formatierung der Kapitelüberschriften %%%%%%%%%%%%%%%%%%
\makechapterstyle{diss}{
  \renewcommand{\printchaptername}{}
  \renewcommand{\chapternamenum}{}
  \renewcommand{\chaptitlefont}{\LARGE}
  \renewcommand{\chapnumfont}{\chaptitlefont}
  \renewcommand{\printchapternum}{\chapnumfont \thechapter\space}
  \renewcommand{\afterchapternum}{}
}
\chapterstyle{diss}
%%%%%%%%%%%%%%%% Formatierung der Sections u.s.w. %%%%%%%%%%%%%%%%%%%
\setsecnumformat{\csname the#1\endcsname \hspace{1ex}}
\setsecheadstyle{\Large}
\setsubsecheadstyle{\large}
\setsubsubsecheadstyle{\normalfont}
\setparaheadstyle{\normalfont\raggedright}
%%%%%%%%%%% Abstände vor Kapitel etc. %%%
\setlength{\beforechapskip}{2\baselineskip}
\setlength{\midchapskip}{.5\baselineskip}
\setlength{\afterchapskip}{\baselineskip}
\setbeforesecskip{2\baselineskip}
\setbeforeparaskip{0\baselineskip}
\setaftersecskip{\baselineskip}
%%%%%%%%%%%%%% zweispaltiger Text %%%%%%%%%%%%%%%%%
\usepackage{parallel}
%%%% Verwendung \LR{linke Spalte}{rechte Spalte}
\newcommand\ZLR[2]{%
	\vspace{0.5\baselineskip}
	\begin{Parallel}{}{}
    \ParallelLText{\small\noindent #1}%
    \ParallelRText{\small\noindent #2}%
    \ParallelPar
  	\end{Parallel}
	\vspace{0.5\baselineskip}%
} 
%%%%%%% Fußnoten %%%%%%%%%%%%%%%%%%
\setlength{\footmarkwidth}{0em} 
\setlength{\footmarksep}{0em}
%%%%%%%%%%%% Indices erstellen %%%%%%%%%%%%%%%%%%%%
\usepackage[nonewpage]{imakeidx}
\makeindex[name=antik,title=Index der antiken Autoren,intoc=true,columns=1]
\newcommand{\stelle}[3]{#1, #2 #3\aindex{#1!#2!#3}} % Befehl zur Angabe von Stellen im Index der antiken Autoren: #1 Autor, #2 Buch, #3 Kapitel,Paragraph
\newcommand{\aindex}{\index[antik]}
\makeindex[name=bibel,title=Index der Bibelstellen,intoc=true,columns=1]
\usepackage[RGG]{bibleref-german} % Paket zur Angabe und Indizierung von Bibelstellen
\newcommand{\bibel}[2]{\ibibleverse{#1}(#2)} % #1 Buch #2 Stelle (Muster Kapitel:Vers)
\newcommand{\bindex}{\index[bibel]}
\renewcommand{\biblerefindex}{\index[bibel]}
\makeatletter
\brg@providecs{br@ITim}{\br@ITimothy}% Nötige Korrekturen von bibleref-german
\brg@providecs{br@IITim}{\br@IITimothy}%
\brg@providecs{br@1Tim}{\br@ITimothy}%
\brg@providecs{br@2Tim}{\br@IITimothy}%
\makeatother
\biblerefmap{Gen}{01} % Für die richtige Sortierung des Index
\biblerefmap{Ex}{02}
\biblerefmap{Lev}{03}
\biblerefmap{Num}{04}
\biblerefmap{Dt}{05}

\biblerefmap{Jos}{06}
\biblerefmap{Ri}{07}
\biblerefmap{Ru}{08}
\biblerefmap{1Sam}{09}
\biblerefmap{2Sam}{11}
\biblerefmap{1Kön}{11}
\biblerefmap{2Kön}{12}
\biblerefmap{1Chr}{13}
\biblerefmap{2Chr}{14}
\biblerefmap{Esr}{15}
\biblerefmap{Neh}{16}
\biblerefmap{Est}{17}
\biblerefmap{Tob}{18}
\biblerefmap{Jdt}{19}
\biblerefmap{1Makk}{20}
\biblerefmap{2Makk}{21}

\biblerefmap{Hi}{22}
\biblerefmap{Ps}{23}
\biblerefmap{Spr}{24}
\biblerefmap{Koh}{25}
\biblerefmap{Hld}{26}
\biblerefmap{Weish}{27}
\biblerefmap{Sir}{28}

\biblerefmap{Jes}{29}
\biblerefmap{Jer}{30}
\biblerefmap{Klgl}{31}
\biblerefmap{Bar}{32}
\biblerefmap{Hes}{33}
\biblerefmap{Dan}{34}
\biblerefmap{Hos}{35}
\biblerefmap{Joel}{36}
\biblerefmap{Am}{37}
\biblerefmap{Obd}{38}
\biblerefmap{Jon}{39}
\biblerefmap{Mi}{40}
\biblerefmap{Nah}{41}
\biblerefmap{Hab}{42}
\biblerefmap{Zef}{43}
\biblerefmap{Hag}{44}
\biblerefmap{Sach}{45}
\biblerefmap{Mal}{46}

\biblerefmap{Mt}{47}
\biblerefmap{Mk}{48}
\biblerefmap{Lk}{49}
\biblerefmap{Joh}{50}
\biblerefmap{Apg}{51}
\biblerefmap{Röm}{52}
\biblerefmap{1Kor}{53}
\biblerefmap{2Kor}{54}
\biblerefmap{Gal}{55}
\biblerefmap{Eph}{56}
\biblerefmap{Phil}{57}
\biblerefmap{Kol}{58}
\biblerefmap{1Thess}{59}
\biblerefmap{2Thess}{60}
\biblerefmap{1Tim}{61}
\biblerefmap{2Tim}{62}
\biblerefmap{Tit}{63}
\biblerefmap{Phlm}{64}
\biblerefmap{Hebr}{65}
\biblerefmap{Jak}{66}
\biblerefmap{1Petr}{67}
\biblerefmap{2Petr}{68}
\biblerefmap{1Joh}{69}
\biblerefmap{2Joh}{70}
\biblerefmap{3Joh}{71}
\biblerefmap{Jud}{72}
\biblerefmap{Apk}{73}
\makeindex[name=autoren,title=Index der modernen Autoren,intoc=true,columns=1] % Index der modernen Autoren
\newcommand{\autindex}{\index[autoren]}
\indexsetup{level=\section*,toclevel=section,headers={\indexname}{\indexname}}
%%%%%% Hyperref und Titelangaben
\usepackage[implicit=true,plainpages=false,hyperindex=true,pdfpagelabels,pdfa,pdfusetitle]{hyperref}
\hypersetup{%
   pdftitle = {TITEL},
   pdfsubject = {THEMA},
   pdfkeywords = {SCHLAGWORTE},
   pdfauthor = {\textcopyright\ AUTOR},
}
\usepackage{longtable,booktabs} % Tabellen
%%%%%%% Zum Schluß noch Hebräisch (wegen bidi)
\usepackage{bidi}
\newfontfamily\hebrewfont[Mapping=tex-text,Script=Greek,Ligatures=Common]{Ezra
  SIL}
\newcommand{\texthebrew}[1]{\RL{\hebrewfont #1}}
\newenvironment{hebrew}{\setRL\hebrewfont}{\setLR}
%%%%%%%%%%%%%%%%%%%%%%%%%%%%%%%%%%%%%%%%%%%%%%%%
%%%%%%%%%%%%%%%%%%%%%%%%%%%%%%%%%%%%%%%%%%%%%%%%
%%%%%%%%%%%%%%%%%%%%%%%%%%%%%%%%%%%%%%%%%%%%%%%%
% Dokumentanfang
\begin{document}
%%%%%%%%%%%%%%%%%% TITELEI %%%%%%%%%%%%%%%%%%
\begin{titlingpage}
\thispagestyle{empty}
\begin{center}
\Large{AUTOR\\TITEL}\\
\end{center}
\vspace*{\fill}
\begin{center}
\large{JAHR}\par
\end{center}
\end{titlingpage}
%%%%%%%%%%%%%%%%%% VORWORT %%%%%%%%%%%%%%%%%%%%%
\frontmatter
\chapter{Vorwort}
% Beispieltext - sollte gelöscht werden
Lorem ipsum dolor sit amet, consectetur adipiscing elit. Sed ut leo sem, eget ornare nunc. Donec gravida eleifend nisi. Vivamus odio turpis, sodales a facilisis ac, tincidunt a massa. Donec eu tortor ac nisl scelerisque congue at quis odio. Maecenas tortor sem, pellentesque quis dapibus eu, tempor sit amet magna. Vivamus eros urna, posuere ullamcorper consequat vitae, molestie quis orci. Ut pellentesque porta justo suscipit volutpat. Sed et quam mauris. Ut non lectus libero, id facilisis enim. Integer et metus mi. Vivamus congue, ante a vulputate ornare, arcu lacus ultricies libero, at volutpat nulla mauris id ipsum. Duis nisl nisl, rhoncus vitae consequat ac, varius at turpis. Vivamus pretium, enim vitae vehicula bibendum, lectus quam egestas diam, ac bibendum erat nunc dapibus erat. Nullam varius viverra metus, at convallis libero interdum eu.

Fusce eget purus urna. Donec eleifend adipiscing orci, ut accumsan erat tincidunt vitae. Mauris rhoncus ligula et felis gravida congue. Donec risus quam, rutrum ac vehicula ac, suscipit ac tellus. Proin feugiat, felis at accumsan consequat, odio enim lobortis purus, eu vehicula purus arcu et lacus. Aenean sodales neque id odio vulputate consectetur molestie purus varius. Etiam a lorem orci, vel vehicula dui. Nullam egestas elementum dictum. Aenean in metus eros. Duis tortor magna, iaculis a pulvinar at, tincidunt ac eros. Suspendisse scelerisque ultrices porta. Vivamus nec neque ut sem iaculis viverra ullamcorper a est. Vivamus metus arcu, mattis eget accumsan mollis, molestie eu sem. Nunc pretium dapibus dictum. Nullam auctor faucibus porttitor.

Vivamus pellentesque quam et dolor ullamcorper rutrum. Donec fringilla nulla sed purus gravida viverra. Pellentesque habitant morbi tristique senectus et netus et malesuada fames ac turpis egestas. Nullam a sapien in est eleifend facilisis nec eget quam. Phasellus aliquet, ipsum sed pulvinar varius, arcu sem imperdiet quam, semper pharetra eros ligula nec lectus. Nam facilisis accumsan metus, id porta odio vehicula sit amet. Duis adipiscing auctor tortor, porttitor eleifend ligula vulputate vel. Cum sociis natoque penatibus et magnis dis parturient montes, nascetur ridiculus mus. Etiam scelerisque tempor nulla ac rhoncus. Integer metus nisi, vehicula sit amet scelerisque sed, tempor sed neque. Aenean erat orci, fermentum tempus accumsan ut, semper sit amet augue. Nam neque est, auctor ultrices dapibus eget, placerat sit amet odio. Nulla facilisi. Proin vitae dolor ipsum, eu euismod massa. Praesent sit amet risus in mauris consequat condimentum eget ut nulla.

Aenean auctor mauris vel est gravida varius. Nunc sapien justo, vestibulum sed convallis ac, varius at ipsum. Sed tincidunt neque ac dolor laoreet quis convallis lectus consectetur. Sed at pharetra ante. Donec ornare, nibh ac euismod placerat, nulla lacus volutpat justo, eget dictum arcu nisl sed sem. Quisque tempor ligula ut nulla porttitor viverra. Donec non arcu id velit blandit ultrices. Mauris leo nulla, blandit sit amet elementum ut, iaculis nec mi. Etiam rutrum faucibus massa in rhoncus. Duis fermentum, felis id eleifend posuere, justo neque sagittis massa, quis dignissim risus velit quis risus.

Etiam rutrum elit a diam dignissim facilisis. In tincidunt, erat at placerat eleifend, ipsum risus feugiat arcu, et imperdiet odio est commodo mauris. Vivamus commodo lacinia eleifend. Duis massa mi, tempus nec gravida in, faucibus sit amet purus. Integer ac arcu sed libero mollis condimentum vitae sit amet arcu. Donec rhoncus tincidunt sem ac ornare. Nullam a lectus tortor, ac euismod neque. Vivamus nisi leo, iaculis interdum fermentum a, hendrerit ac enim. Aenean vulputate lorem feugiat dui blandit vel interdum sapien euismod. Cras laoreet aliquam urna ac tempor. Aliquam et diam vel felis rutrum dignissim sit amet convallis urna. Integer convallis condimentum placerat. 
% Hier endet der Beispieltext
%%%%%%%%%%%%%%%%%% INHALTSVERZEICHNIS %%%%%%%%%%%%
\cleardoublepage
\renewcommand{\contentsname}{Inhalt}
\tableofcontents*
%%%%%%%%%%%%%%%%%% HAUPTTEXT %%%%%%%%%%%%%%%%%%
\mainmatter
\chapter{Ein Kapitel}
% Beispieltext - sollte gelöscht werden
%%%%%%%%%%%%%%%%%%%%%%%%%%%%%%%%%%%%%%%%%%
\section{Ein Unterkapitel}
Lorem ipsum dolor sit amet, consectetur adipiscing elit. Sed ut leo sem, eget ornare nunc. Donec gravida eleifend nisi. Vivamus odio turpis, sodales a facilisis ac, tincidunt a massa. Donec eu tortor ac nisl scelerisque congue at quis odio. Maecenas tortor sem, \enquote{pellentesque \enquote{quis} dapibus eu}, tempor sit amet magna. Vivamus eros urna, posuere ullamcorper consequat vitae, molestie quis orci. Ut pellentesque porta justo suscipit volutpat. Sed et quam mauris. Ut non lectus libero, id facilisis enim. Integer et metus mi. Vivamus congue, ante a vulputate ornare, arcu lacus ultricies libero, at volutpat nulla mauris id ipsum. Duis nisl nisl, rhoncus vitae consequat ac, varius at turpis. Vivamus pretium, enim vitae vehicula bibendum, lectus quam egestas diam, ac bibendum erat nunc dapibus erat. Nullam varius viverra metus, at convallis libero interdum eu.
% keine Leerzeile
\blockquote{Fusce eget purus urna. Donec eleifend adipiscing orci, ut accumsan erat tincidunt vitae. Mauris rhoncus ligula et felis gravida congue. Donec risus quam, rutrum ac vehicula ac, suscipit ac tellus. Proin feugiat, felis at accumsan consequat, odio enim lobortis purus, eu vehicula purus arcu et lacus. Aenean sodales neque id odio vulputate consectetur molestie purus varius.\footnote{Vgl. \bibel{Gen}{1:1}} Etiam a lorem orci, vel vehicula dui. Nullam egestas elementum dictum. Aenean in metus eros. Duis tortor magna, iaculis a pulvinar at, tincidunt ac eros. Suspendisse scelerisque ultrices porta. Vivamus nec neque ut sem iaculis viverra ullamcorper a est. Vivamus metus arcu, mattis eget accumsan mollis, molestie eu sem. Nunc pretium dapibus dictum. Nullam auctor faucibus porttitor.}
% keine Leerzeile
Vivamus \blockquote{pellentesque quam et dolor ullamcorper rutrum}.\footnote{Der Befehl blockquote verhält sich je nach länge des markierten Textes unterschiedlich!} Donec fringilla nulla sed purus gravida viverra. Pellentesque habitant morbi tristique senectus et netus et malesuada fames ac turpis egestas. Nullam a sapien in est eleifend facilisis nec eget quam. Phasellus aliquet, ipsum sed pulvinar varius, arcu sem imperdiet quam, semper pharetra eros ligula nec lectus. Nam facilisis accumsan metus, id porta odio vehicula sit amet. Duis adipiscing auctor tortor, porttitor eleifend ligula vulputate vel.\footnote{Cum sociis natoque penatibus et magnis dis parturient montes, nascetur ridiculus mus.} Etiam scelerisque tempor nulla ac rhoncus. Integer metus nisi, vehicula sit amet scelerisque sed, tempor sed neque. Aenean erat orci, fermentum tempus accumsan ut, semper sit amet augue.\footnote{Vgl. \stelle{Ath.}{inc.}{12,32}.} Nam neque est, auctor ultrices dapibus eget, placerat sit amet odio. Nulla facilisi. Proin vitae dolor ipsum, eu euismod massa. Praesent sit amet risus in mauris consequat condimentum eget ut nulla.
%%%%%%%%%%%%%%%%%%%%%%%%%%%%%%%%%%%%%%%%%%%
\section{Noch ein Unterkapitel}
Aenean auctor mauris vel est gravida varius. Nunc sapien justo, vestibulum sed convallis ac, varius at ipsum. Sed tincidunt neque ac dolor laoreet quis convallis lectus consectetur. Sed at pharetra ante. Donec ornare, nibh ac euismod placerat, nulla lacus volutpat justo, eget dictum arcu nisl sed sem. Quisque tempor ligula ut nulla porttitor viverra.\footcite[50]{bizer_armenische_1969} Donec non arcu id velit blandit ultrices. Mauris leo nulla, blandit sit amet elementum ut, iaculis nec mi. Etiam rutrum faucibus massa in rhoncus. Duis fermentum, felis id eleifend posuere,\footnote{Vgl. \bibel{1Kor}{15:3}} justo neque sagittis massa, quis dignissim risus velit quis risus.
% keine Leerzeile
\ZLR{% linke Spalte
Etiam rutrum elit a diam dignissim facilisis. In tincidunt, erat at placerat eleifend, ipsum risus feugiat arcu, et imperdiet odio est commodo mauris. Vivamus commodo lacinia eleifend. Duis massa mi, tempus nec gravida in, faucibus sit amet purus. Integer ac arcu sed libero mollis condimentum vitae sit amet arcu. Donec rhoncus tincidunt sem ac ornare. Nullam a lectus tortor, ac euismod neque. Vivamus nisi leo, iaculis interdum fermentum a, hendrerit ac enim. Aenean vulputate lorem feugiat dui blandit vel interdum sapien euismod. Cras laoreet aliquam urna ac tempor. Aliquam et diam vel felis rutrum dignissim sit amet convallis urna. Integer convallis condimentum placerat.}%
{% rechte Spalte
Etiam rutrum elit a diam dignissim facilisis. In tincidunt, erat at placerat eleifend, ipsum risus feugiat arcu, et imperdiet odio est commodo mauris. Vivamus commodo lacinia eleifend. Duis massa mi, tempus nec gravida in, faucibus sit amet purus. Integer ac arcu sed libero mollis condimentum vitae sit amet arcu. Donec rhoncus tincidunt sem ac ornare. Nullam a lectus tortor, ac euismod neque. Vivamus nisi leo, iaculis interdum fermentum a, hendrerit ac enim. Aenean vulputate lorem feugiat dui blandit vel interdum sapien euismod. Cras laoreet aliquam urna ac tempor. Aliquam et diam vel felis rutrum dignissim sit amet convallis urna. Integer convallis condimentum placerat.}
%% keine Leerzeile
Aenean\autindex{Aeneas} auctor mauris vel est gravida varius. Nunc sapien justo, vestibulum sed convallis ac, varius at ipsum. Sed tincidunt neque ac dolor laoreet quis convallis lectus consectetur. Sed at pharetra ante. Donec ornare, nibh ac euismod placerat, nulla lacus volutpat justo, eget dictum arcu nisl sed sem. Quisque tempor ligula ut nulla porttitor viverra. Donec non arcu id velit blandit ultrices. Mauris leo nulla, blandit sit amet elementum ut, iaculis nec mi. Etiam rutrum faucibus massa in rhoncus. Duis fermentum, felis id eleifend posuere, justo\footnote{Interessante Beobachtungen von Maier.\autindex{Maier}} neque sagittis massa, quis dignissim risus velit quis risus.
%%%%%%%%%%%%%%%%%%%%%%%%%%%%%%%%%%%%%%%%%%%%%
\section{Ein Unterkapitel mit griechischem Text}
ὥστε ᾁδοντες ἠμέλησαν σίτων τε καὶ ποτῶν, καὶ ἔλαθον τελευτή-σαντες αὑτούς: ἐξ ὧν τὸ τεττίγων γένος μετ᾽ ἐκεῖνο φύεται, γέρας τοῦτο παρὰ Μουσῶν λαβόν, μηδὲν τροφῆς δεῖσθαι γενόμενον, ἀλλ᾽ ἄσιτόν τε καὶ ἄποτον εὐθὺς ᾁδειν, ἕως ἂν τελευτήσῃ, καὶ μετὰ ταῦτα ἐλθὸν παρὰ μούσας ἀπαγγέλλειν τίς τίνα αὐτῶν τιμᾷ τῶν ἐνθάδε. Τερψιχόρᾳ μὲν οὖν τοὺς ἐν τοῖς χοροῖς τετιμηκότας αὐτὴν ἀπαγγέλλοντες.\footnote{Vgl. \stelle{Ath.}{decr.}{1,2}.}
 \subsection{Und jetzt noch eine Tabelle}
\begin{longtable}{p{.15\textwidth}p{.75\textwidth}} % Zwei Spalten 15% und 75% der Textbreite breit
	\hline
	Kap. &  Inhalt \\
	\hline
	\endhead % Ende der Kopfzeile
	\hline
	\endfoot % Ende der Fußzeile
	1 & Blablub \\ %Tabelleninhalt
	2 & Blablub \\
	3 & Blablub \\		
	4 & Blablub \\
\end{longtable}
\section{Und ein Unterkapitel mit hebräischem Text}
Etiam rutrum elit a diam dignissim facilisis. In tincidunt, erat at
placerat eleifend, ipsum risus feugiat arcu, et imperdiet odio est
commodo mauris. \texthebrew{בְּרֵאשִׁ֖ית בָּרָ֣א אֱלֹהִ֑ים אֵ֥ת הַשָּׁמַ֖יִם וְאֵ֥ת
  הָאָֽרֶץ׃}\footnote{\bibel{Gen}{1:1}.}  Vivamus commodo lacinia
eleifend.

\begin{hebrew}
בְּרֵאשִׁ֖ית בָּרָ֣א אֱלֹהִ֑ים אֵ֥ת הַשָּׁמַ֖יִם וְאֵ֥ת הָאָֽרֶץ׃

2וְהָאָ֗רֶץ הָיְתָ֥ה תֹ֨הוּ֙ וָבֹ֔הוּ וְחֹ֖שֶׁךְ עַל־פְּנֵ֣י תְהֹ֑ום וְר֣וּחַ אֱלֹהִ֔ים מְרַחֶ֖פֶת עַל־פְּנֵ֥י הַמָּֽיִם׃

3וַיֹּ֥אמֶר אֱלֹהִ֖ים יְהִ֣י אֹ֑ור וַֽיְהִי־אֹֽור׃

4וַיַּ֧רְא אֱלֹהִ֛ים אֶת־הָאֹ֖ור כִּי־טֹ֑וב וַיַּבְדֵּ֣ל אֱלֹהִ֔ים בֵּ֥ין הָאֹ֖ור וּבֵ֥ין הַחֹֽשֶׁךְ׃

5וַיִּקְרָ֨א אֱלֹהִ֤ים׀ לָאֹור֙ יֹ֔ום וְלַחֹ֖שֶׁךְ קָ֣רָא לָ֑יְלָה וַֽיְהִי־עֶ֥רֶב וַֽיְהִי־בֹ֖קֶר יֹ֥ום אֶחָֽד׃

\end{hebrew}

 Vivamus commodo lacinia eleifend.
% Hier endet der Beispieltext % Verlinkung der Unterkapitel
%%%%%%%%%%%%%%%%%%%%%%%%%%%%%%%%%%%%%%%%%%%%
%%%%%%%%%%%%%%%% ANHANG  %%%%%%%%%%%
%%%%%%%%%%%%%%%%%%%%%%%%%%%%%%%%%%%%%%%%%%%%
\backmatter
%%% Literaturverzeichnis
\nocite{*} % Alle Titel in der Bibtex-Datei werden im Literaturverzeichnis angeführt, ggfs. löschen
\printbibliography % Druckt das Literaturverzeichnis aus
%%% Indices
\chapter{Indices}
\printindex[bibel]
\printindex[antik]
\printindex[autoren]
% Dokumentende
\end{document}
%%% Local Variables:
%%% mode: latex
%%% coding: utf-8-unix
%%% TeX-master: t
%%% End:
